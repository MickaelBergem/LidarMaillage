\documentclass[12pt , a4paper]{article}% , twoside , openright

% Modification des longueurs, pour corriger le Twoside
%\setlength{\oddsidemargin}{52pt}
%\setlength{\evensidemargin}{10pt}
\usepackage{graphicx}
\usepackage{float}


\usepackage[francais]{babel}
\usepackage[T1]{fontenc}
\usepackage[utf8]{inputenc}
\usepackage{amsfonts}
\usepackage{amsmath}
\usepackage{graphicx}
\usepackage{eurosym}
\usepackage{yfonts}
\usepackage[dvipsnames]{xcolor}
	\definecolor{mygreen}{rgb}{0,0.6,0}
	\definecolor{mygray}{rgb}{0.5,0.5,0.5}
	\definecolor{mymauve}{rgb}{0.58,0,0.82}
\usepackage{listings}
\lstdefinestyle{python}{ %
    language=Python,
    backgroundcolor=\color{black!5}, % set backgroundcolor
    basicstyle=\footnotesize,% basic font setting
    commentstyle=\color{red},
    keywordstyle=\color{orange},
    stringstyle=\color{mygreen},
    breaklines=true,
    breakatwhitespace=true
}
\usepackage[francais]{layout}
\usepackage{pifont}
\usepackage{fancyhdr}
\usepackage[hidelinks,urlcolor=blue,linkcolor=black,citecolor=red,linktoc=all,colorlinks=true,bookmarks=true]{hyperref}
\usepackage{multirow}
\usepackage{textcomp}
\usepackage{xtab}
\usepackage[]{algorithm2e}

%\setlength{\variable}{dimension}	
%\newcommand{\commande}[nops]{suite de commandes avec #nb}
%\addcontentsline{toc}{chapter}{Introduction}
%environnement lstlisting pour le code C++.

\graphicspath{{Illustrations/}}
\lstset{language=C++}

\pagestyle{fancy} % headings va surcharger tous les entetes et pieds de page, plain met juste le num�ro de paage en bas.

\fancyhf{}
\fancyhead[R]{\thepage}
\fancyhead[L]{\leftmark}

\renewcommand{\sectionmark}[1]{% #1 contient le titre de la section
\markright{\thesection\ #1}}
