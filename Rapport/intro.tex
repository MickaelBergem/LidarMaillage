\newpage
\thispagestyle{plain}

\vspace*{\stretch{1}}



\section*{\centering Introduction}
\addcontentsline{toc}{section}{Introduction}
Dans le cadre du cours de Maillages et Applications de l'École des Ponts ParisTech, j'ai choisi le projet ``modélisation de bâtiments en 3D sous forme d’un maillage dense à partir d’un nuage de points Lidar''. J'ai du développer un algorithme qui, à partir d'un nuage de points Lidar, produit un maillage dense de l'environnement urbain duquel le nuage est issu.

Contrairement aux autres groupes, j'ai travaillé seul (nous étions un nombre impair) sur ce projet orienté recherche, et j'ai bénéficié des conseils de Florent Lafarge pour me guider sur la méthode à adopter.

J'ai utilisé les fonctions de traitements des maillages de la librairie CGAL, et développé un algorithme en C++.
\vspace*{\stretch{2}}